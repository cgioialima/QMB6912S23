\documentclass[11pt]{article}
\usepackage{fullpage}
\usepackage{palatino}
\usepackage{amsfonts,amsmath,amssymb}
% \usepackage{graphicx}

\usepackage[margin=0.25in]{geometry}

\usepackage{listings}
\usepackage{textcomp}
\usepackage{color}
\usepackage{longtable}

\definecolor{dkgreen}{rgb}{0,0.6,0}
\definecolor{gray}{rgb}{0.5,0.5,0.5}
\definecolor{mauve}{rgb}{0.58,0,0.82}

\lstset{frame=tb,
  language=R,
  aboveskip=3mm,
  belowskip=3mm,
  showstringspaces=false,
  columns=flexible,
  basicstyle={\small\ttfamily},
  numbers=none,
  numberstyle=\tiny\color{gray},
  keywordstyle=\color{blue},
  commentstyle=\color{dkgreen},
  stringstyle=\color{mauve},
  breaklines=true,
  breakatwhitespace=true,
  tabsize=3
}



\ifx\pdftexversion\undefined
    \usepackage[dvips]{graphicx}
\else
    \usepackage[pdftex]{graphicx}
    \usepackage{epstopdf}
    \epstopdfsetup{suffix=}
\fi

\usepackage{subfig}


% This allows pdflatex to print the curly quotes in the
% significance codes in the output of the GAM.
\UseRawInputEncoding

\begin{document}

%%%%%%%%%%%%%%%%%%%%%%%%%%%%%%%%%%%%%%%%
% Problem Set 7
%%%%%%%%%%%%%%%%%%%%%%%%%%%%%%%%%%%%%%%%

\pagestyle{empty}
{\noindent\bf Spring 2023 \hfill Carolina.~Lima}
\vskip 16pt
\centerline{\bf University of Central Florida}
\centerline{\bf College of Business}
\vskip 16pt
\centerline{\bf QMB 6911}
\centerline{\bf Capstone Project in Business Analytics}
\vskip 10pt
\centerline{\bf Solutions:  Problem Set \#7}
\vskip 32pt
\noindent
% 
\section{Data Description}

This analysis follows the script \texttt{assignment7.R} to produce a more accurate model for used Home prices with the data from \texttt{HomeSales.dat} in the \texttt{Data} folder. 
The dataset includes the following variables.

\begin{table}[ht]
\centering
\begin{tabular}{ll}
  \hline
    Variable & Definition \\
  \hline
    {\tt year\_built}      & the year in which the house was constructed \cr
    {\tt num\_beds}      & the number of bedrooms in the house \cr
    {\tt num\_baths}      & the number of bathrooms in the house\cr
    {\tt floor\_space}      & the area of floor space in the house, in square feet\cr
    {\tt lot\_size}      & the area of lot on which the house was built, in square feet \cr
    {\tt has\_garage}      & an indicator for whether the house has a garage \cr
    {\tt has\_encl\_patio}      & an indicator for whether the house has an enclosed patio \cr
    {\tt has\_security\_gate}      & an indicator for whether the property is accessed through a security gate \cr
    {\tt has\_pool}      & an indicator for whether the property includes a pool \cr
    {\tt transit\_score}      & an integer to represent the convenience of transportation options \cr
    {\tt school\_score}      & an integer to represent the quality of the schools in the county  \cr
    {\tt type\_of\_buyer}      & a categorical variable to indicate the type of buyer, \cr
		& either ``Owner-Occupied'' or ``Rental''  \cr
    {\tt price}      & the price at which the home was sold \cr
  \hline
\end{tabular}
\end{table}

\noindent Two new variables were created:

\begin{table}[ht]
\centering
\begin{tabular}{ll}
  \hline
    Variable & Definition \\
  \hline
    {\tt log\_price}      & logarithm of the house price \cr
    {\tt age}      & the age of the house, calculated based on the year built \cr

  \hline
\end{tabular}
\end{table}
%
I will first estimate a model with nonlinear functional forms, and then consider exclusions of insignificant variables from the full model, with nonlinearity taken into account. 
This approach allows for inclusion of possibly irrelevant variables and avoids excluding any relevant variables. 

%%%%%%%%%%%%%%%%%%%%%%%%%%%%%%%%%%%%%%%%
% Example with a Categorical Numerical Variable
%%%%%%%%%%%%%%%%%%%%%%%%%%%%%%%%%%%%%%%%
\section{Example with a Categorical Numerical Variable}
\subsection{School Score}
Decided to try the variable school score as a factor, differently than what it was used on Problem Set \#6.
Table \ref{tab:fact.school} shows odel 1 being the full model, model 2 the best reduced model found in Problem Set \#6 and model 3 the full model using school score as a factor. We see there is no difference in R squared for any of these models, however school schore goes from being insiginifcant is model 1 to having each different factor being significant in model 3. Floor space continues to be statistically insiginificant in both models.


\begin{center}
\begin{longtable}{l c c c}
\hline
 & Model 1 & Model 2 & Model 3 \\
\hline
\endfirsthead
\hline
 & Model 1 & Model 2 & Model 3 \\
\hline
\endhead
\hline
\endfoot
\hline
\multicolumn{4}{l}{\scriptsize{$^{***}p<0.001$; $^{**}p<0.01$; $^{*}p<0.05$}}\\
\caption{Dollar Value of Home Prices}
\label{tab:fact.school}
\endlastfoot \\
(Intercept)                & $12.22^{***}$ & $12.28^{***}$ & $12.16^{***}$ \\
                           & $(0.06)$      & $(0.04)$      & $(0.07)$      \\
as.factor(num\_beds)2      & $0.08^{**}$   & $0.09^{**}$   & $0.07^{*}$    \\
                           & $(0.03)$      & $(0.03)$      & $(0.03)$      \\
as.factor(num\_beds)3      & $0.20^{***}$  & $0.23^{***}$  & $0.19^{***}$  \\
                           & $(0.04)$      & $(0.04)$      & $(0.04)$      \\
as.factor(num\_beds)4      & $0.42^{***}$  & $0.46^{***}$  & $0.41^{***}$  \\
                           & $(0.05)$      & $(0.04)$      & $(0.05)$      \\
as.factor(num\_beds)6      & $0.48^{***}$  & $0.54^{***}$  & $0.47^{***}$  \\
                           & $(0.08)$      & $(0.07)$      & $(0.08)$      \\
as.factor(num\_beds)8      & $0.83^{***}$  & $0.92^{***}$  & $0.83^{***}$  \\
                           & $(0.10)$      & $(0.07)$      & $(0.10)$      \\
as.factor(num\_baths)2     & $0.07^{***}$  & $0.08^{***}$  & $0.07^{***}$  \\
                           & $(0.02)$      & $(0.02)$      & $(0.02)$      \\
as.factor(num\_baths)3     & $0.09^{*}$    & $0.10^{**}$   & $0.09^{**}$   \\
                           & $(0.03)$      & $(0.03)$      & $(0.03)$      \\
floor\_space               & $0.00$        &               & $0.00$        \\
                           & $(0.00)$      &               & $(0.00)$      \\
lot\_size                  & $0.00^{***}$  & $0.00^{***}$  & $0.00^{***}$  \\
                           & $(0.00)$      & $(0.00)$      & $(0.00)$      \\
has\_garage                & $0.28^{***}$  & $0.31^{***}$  & $0.28^{***}$  \\
                           & $(0.03)$      & $(0.02)$      & $(0.03)$      \\
has\_encl\_patio           & $0.05^{**}$   & $0.05^{**}$   & $0.05^{**}$   \\
                           & $(0.01)$      & $(0.01)$      & $(0.01)$      \\
has\_security\_gate        & $0.23^{***}$  & $0.23^{***}$  & $0.23^{***}$  \\
                           & $(0.02)$      & $(0.02)$      & $(0.02)$      \\
has\_pool                  & $0.10^{***}$  & $0.10^{***}$  & $0.10^{***}$  \\
                           & $(0.03)$      & $(0.03)$      & $(0.03)$      \\
transit\_score             & $0.07^{***}$  & $0.07^{***}$  & $0.07^{***}$  \\
                           & $(0.00)$      & $(0.00)$      & $(0.00)$      \\
school\_score              & $0.00$        &               &               \\
                           & $(0.00)$      &               &               \\
type\_of\_buyerRental      & $0.11^{***}$  & $0.11^{***}$  & $0.11^{***}$  \\
                           & $(0.02)$      & $(0.02)$      & $(0.02)$      \\
age                        & $-0.01^{***}$ & $-0.01^{***}$ & $-0.01^{***}$ \\
                           & $(0.00)$      & $(0.00)$      & $(0.00)$      \\
as.factor(school\_score)2  &               &               & $0.10^{**}$   \\
                           &               &               & $(0.03)$      \\
as.factor(school\_score)3  &               &               & $0.10^{**}$   \\
                           &               &               & $(0.03)$      \\
as.factor(school\_score)4  &               &               & $0.09^{**}$   \\
                           &               &               & $(0.03)$      \\
as.factor(school\_score)5  &               &               & $0.09^{**}$   \\
                           &               &               & $(0.03)$      \\
as.factor(school\_score)6  &               &               & $0.10^{**}$   \\
                           &               &               & $(0.03)$      \\
as.factor(school\_score)7  &               &               & $0.10^{**}$   \\
                           &               &               & $(0.03)$      \\
as.factor(school\_score)8  &               &               & $0.10^{**}$   \\
                           &               &               & $(0.03)$      \\
as.factor(school\_score)9  &               &               & $0.09^{*}$    \\
                           &               &               & $(0.04)$      \\
as.factor(school\_score)10 &               &               & $0.10^{*}$    \\
                           &               &               & $(0.04)$      \\
\hline
R$^2$                      & $0.65$        & $0.65$        & $0.65$        \\
Adj. R$^2$                 & $0.65$        & $0.65$        & $0.65$        \\
Num. obs.                  & $1862$        & $1862$        & $1862$        \\
\end{longtable}
\end{center}

%\pagebreak

\subsection{Transit Score}
Separated variable for transit score in 4 different categories to analyze significance in the model.
Table \ref{tab:cat.transit} shows model 1 being the full model with school score as a factor and model 2 has the adtional separate categories for transit score.
None of the new catecories are statistically significant while the original transit score variable is. There is also no diffrence on R squared and adjusted R squared
New categories will not be added to the model.


\begin{center}
\begin{longtable}{l c c}
\hline
 & Model 1 & Model 2 \\
\hline
\endfirsthead
\hline
 & Model 1 & Model 2 \\
\hline
\endhead
\hline
\endfoot
\hline
\multicolumn{3}{l}{\scriptsize{$^{***}p<0.001$; $^{**}p<0.01$; $^{*}p<0.05$}}\\
\caption{Dollar Value of Home Prices}
\label{tab:cat.transit}
\endlastfoot \\
(Intercept)                & $12.16^{***}$ & $12.19^{***}$ \\
                           & $(0.07)$      & $(0.07)$      \\
as.factor(num\_beds)2      & $0.07^{*}$    & $0.07^{*}$    \\
                           & $(0.03)$      & $(0.03)$      \\
as.factor(num\_beds)3      & $0.19^{***}$  & $0.19^{***}$  \\
                           & $(0.04)$      & $(0.04)$      \\
as.factor(num\_beds)4      & $0.41^{***}$  & $0.40^{***}$  \\
                           & $(0.05)$      & $(0.05)$      \\
as.factor(num\_beds)6      & $0.47^{***}$  & $0.46^{***}$  \\
                           & $(0.08)$      & $(0.08)$      \\
as.factor(num\_beds)8      & $0.83^{***}$  & $0.81^{***}$  \\
                           & $(0.10)$      & $(0.10)$      \\
as.factor(num\_baths)2     & $0.07^{***}$  & $0.08^{***}$  \\
                           & $(0.02)$      & $(0.02)$      \\
as.factor(num\_baths)3     & $0.09^{**}$   & $0.09^{**}$   \\
                           & $(0.03)$      & $(0.03)$      \\
floor\_space               & $0.00$        & $0.00$        \\
                           & $(0.00)$      & $(0.00)$      \\
lot\_size                  & $0.00^{***}$  & $0.00^{***}$  \\
                           & $(0.00)$      & $(0.00)$      \\
has\_garage                & $0.28^{***}$  & $0.27^{***}$  \\
                           & $(0.03)$      & $(0.03)$      \\
has\_encl\_patio           & $0.05^{**}$   & $0.05^{**}$   \\
                           & $(0.01)$      & $(0.01)$      \\
has\_security\_gate        & $0.23^{***}$  & $0.23^{***}$  \\
                           & $(0.02)$      & $(0.02)$      \\
has\_pool                  & $0.10^{***}$  & $0.10^{***}$  \\
                           & $(0.03)$      & $(0.03)$      \\
transit\_score             & $0.07^{***}$  & $0.05^{***}$  \\
                           & $(0.00)$      & $(0.01)$      \\
as.factor(school\_score)2  & $0.10^{**}$   & $0.10^{**}$   \\
                           & $(0.03)$      & $(0.03)$      \\
as.factor(school\_score)3  & $0.10^{**}$   & $0.10^{**}$   \\
                           & $(0.03)$      & $(0.03)$      \\
as.factor(school\_score)4  & $0.09^{**}$   & $0.09^{**}$   \\
                           & $(0.03)$      & $(0.03)$      \\
as.factor(school\_score)5  & $0.09^{**}$   & $0.09^{**}$   \\
                           & $(0.03)$      & $(0.03)$      \\
as.factor(school\_score)6  & $0.10^{**}$   & $0.10^{**}$   \\
                           & $(0.03)$      & $(0.03)$      \\
as.factor(school\_score)7  & $0.10^{**}$   & $0.10^{**}$   \\
                           & $(0.03)$      & $(0.03)$      \\
as.factor(school\_score)8  & $0.10^{**}$   & $0.10^{**}$   \\
                           & $(0.03)$      & $(0.03)$      \\
as.factor(school\_score)9  & $0.09^{*}$    & $0.09^{*}$    \\
                           & $(0.04)$      & $(0.04)$      \\
as.factor(school\_score)10 & $0.10^{*}$    & $0.10^{*}$    \\
                           & $(0.04)$      & $(0.04)$      \\
type\_of\_buyerRental      & $0.11^{***}$  & $0.11^{***}$  \\
                           & $(0.02)$      & $(0.02)$      \\
age                        & $-0.01^{***}$ & $-0.01^{***}$ \\
                           & $(0.00)$      & $(0.00)$      \\
ts\_cat3-6                 &               & $0.03$        \\
                           &               & $(0.04)$      \\
ts\_cat6-8                 &               & $0.05$        \\
                           &               & $(0.06)$      \\
ts\_cat8+                  &               & $0.11$        \\
                           &               & $(0.07)$      \\
\hline
R$^2$                      & $0.65$        & $0.65$        \\
Adj. R$^2$                 & $0.65$        & $0.65$        \\
Num. obs.                  & $1862$        & $1862$        \\
\end{longtable}
\end{center}


\subsection{Lot Size}
Transformed lot size into 5 categories. On table \ref{tab:cat_comp} model 1 is our current best model ( full model with school score as factor) and model 2 is the full model with the new lot size categories.
We can see that category  5050-6650 is statistically significant while the others are not, this sugestsa non linear relationship between lot size and the home price. There is no difference on the R squared or adjusted R squared for these models, but since
lot size is a continuous variable, we can do better.
 

\begin{table}
\begin{center}
\begin{tabular}{l c c c c c}
\hline
 & Model 1 & Model 2 & Model 3 & Model 4 & Model 5 \\
\hline
(Intercept)                & $12.22^{***}$ & $12.28^{***}$ & $12.16^{***}$ & $12.19^{***}$ & $12.15^{***}$ \\
                           & $(0.06)$      & $(0.04)$      & $(0.07)$      & $(0.07)$      & $(0.08)$      \\
as.factor(num\_beds)2      & $0.08^{**}$   & $0.09^{**}$   & $0.07^{*}$    & $0.07^{*}$    & $0.07^{*}$    \\
                           & $(0.03)$      & $(0.03)$      & $(0.03)$      & $(0.03)$      & $(0.03)$      \\
as.factor(num\_beds)3      & $0.20^{***}$  & $0.23^{***}$  & $0.19^{***}$  & $0.19^{***}$  & $0.21^{***}$  \\
                           & $(0.04)$      & $(0.04)$      & $(0.04)$      & $(0.04)$      & $(0.05)$      \\
as.factor(num\_beds)4      & $0.42^{***}$  & $0.46^{***}$  & $0.41^{***}$  & $0.40^{***}$  & $0.45^{***}$  \\
                           & $(0.05)$      & $(0.04)$      & $(0.05)$      & $(0.05)$      & $(0.05)$      \\
as.factor(num\_beds)6      & $0.48^{***}$  & $0.54^{***}$  & $0.47^{***}$  & $0.46^{***}$  & $0.53^{***}$  \\
                           & $(0.08)$      & $(0.07)$      & $(0.08)$      & $(0.08)$      & $(0.09)$      \\
as.factor(num\_beds)8      & $0.83^{***}$  & $0.92^{***}$  & $0.83^{***}$  & $0.81^{***}$  & $0.89^{***}$  \\
                           & $(0.10)$      & $(0.07)$      & $(0.10)$      & $(0.10)$      & $(0.11)$      \\
as.factor(num\_baths)2     & $0.07^{***}$  & $0.08^{***}$  & $0.07^{***}$  & $0.08^{***}$  & $0.08^{***}$  \\
                           & $(0.02)$      & $(0.02)$      & $(0.02)$      & $(0.02)$      & $(0.02)$      \\
as.factor(num\_baths)3     & $0.09^{*}$    & $0.10^{**}$   & $0.09^{**}$   & $0.09^{**}$   & $0.09^{*}$    \\
                           & $(0.03)$      & $(0.03)$      & $(0.03)$      & $(0.03)$      & $(0.04)$      \\
floor\_space               & $0.00$        &               & $0.00$        & $0.00$        & $0.00$        \\
                           & $(0.00)$      &               & $(0.00)$      & $(0.00)$      & $(0.00)$      \\
lot\_size                  & $0.00^{***}$  & $0.00^{***}$  & $0.00^{***}$  & $0.00^{***}$  & $0.00^{**}$   \\
                           & $(0.00)$      & $(0.00)$      & $(0.00)$      & $(0.00)$      & $(0.00)$      \\
has\_garage                & $0.28^{***}$  & $0.31^{***}$  & $0.28^{***}$  & $0.27^{***}$  & $0.28^{***}$  \\
                           & $(0.03)$      & $(0.02)$      & $(0.03)$      & $(0.03)$      & $(0.04)$      \\
has\_encl\_patio           & $0.05^{**}$   & $0.05^{**}$   & $0.05^{**}$   & $0.05^{**}$   & $0.03$        \\
                           & $(0.01)$      & $(0.01)$      & $(0.01)$      & $(0.01)$      & $(0.02)$      \\
has\_security\_gate        & $0.23^{***}$  & $0.23^{***}$  & $0.23^{***}$  & $0.23^{***}$  & $0.23^{***}$  \\
                           & $(0.02)$      & $(0.02)$      & $(0.02)$      & $(0.02)$      & $(0.02)$      \\
has\_pool                  & $0.10^{***}$  & $0.10^{***}$  & $0.10^{***}$  & $0.10^{***}$  & $0.11^{***}$  \\
                           & $(0.03)$      & $(0.03)$      & $(0.03)$      & $(0.03)$      & $(0.03)$      \\
transit\_score             & $0.07^{***}$  & $0.07^{***}$  & $0.07^{***}$  & $0.05^{***}$  & $0.07^{***}$  \\
                           & $(0.00)$      & $(0.00)$      & $(0.00)$      & $(0.01)$      & $(0.00)$      \\
school\_score              & $0.00$        &               &               &               &               \\
                           & $(0.00)$      &               &               &               &               \\
type\_of\_buyerRental      & $0.11^{***}$  & $0.11^{***}$  & $0.11^{***}$  & $0.11^{***}$  & $0.13^{***}$  \\
                           & $(0.02)$      & $(0.02)$      & $(0.02)$      & $(0.02)$      & $(0.02)$      \\
age                        & $-0.01^{***}$ & $-0.01^{***}$ & $-0.01^{***}$ & $-0.01^{***}$ & $-0.01^{***}$ \\
                           & $(0.00)$      & $(0.00)$      & $(0.00)$      & $(0.00)$      & $(0.00)$      \\
as.factor(school\_score)2  &               &               & $0.10^{**}$   & $0.10^{**}$   & $0.11^{**}$   \\
                           &               &               & $(0.03)$      & $(0.03)$      & $(0.04)$      \\
as.factor(school\_score)3  &               &               & $0.10^{**}$   & $0.10^{**}$   & $0.08^{*}$    \\
                           &               &               & $(0.03)$      & $(0.03)$      & $(0.03)$      \\
as.factor(school\_score)4  &               &               & $0.09^{**}$   & $0.09^{**}$   & $0.10^{**}$   \\
                           &               &               & $(0.03)$      & $(0.03)$      & $(0.03)$      \\
as.factor(school\_score)5  &               &               & $0.09^{**}$   & $0.09^{**}$   & $0.08^{*}$    \\
                           &               &               & $(0.03)$      & $(0.03)$      & $(0.03)$      \\
as.factor(school\_score)6  &               &               & $0.10^{**}$   & $0.10^{**}$   & $0.08^{*}$    \\
                           &               &               & $(0.03)$      & $(0.03)$      & $(0.04)$      \\
as.factor(school\_score)7  &               &               & $0.10^{**}$   & $0.10^{**}$   & $0.08^{*}$    \\
                           &               &               & $(0.03)$      & $(0.03)$      & $(0.04)$      \\
as.factor(school\_score)8  &               &               & $0.10^{**}$   & $0.10^{**}$   & $0.09^{*}$    \\
                           &               &               & $(0.03)$      & $(0.03)$      & $(0.04)$      \\
as.factor(school\_score)9  &               &               & $0.09^{*}$    & $0.09^{*}$    & $0.09^{*}$    \\
                           &               &               & $(0.04)$      & $(0.04)$      & $(0.04)$      \\
as.factor(school\_score)10 &               &               & $0.10^{*}$    & $0.10^{*}$    & $0.08$        \\
                           &               &               & $(0.04)$      & $(0.04)$      & $(0.05)$      \\
ts\_cat3-6                 &               &               &               & $0.03$        &               \\
                           &               &               &               & $(0.04)$      &               \\
ts\_cat6-8                 &               &               &               & $0.05$        &               \\
                           &               &               &               & $(0.06)$      &               \\
ts\_cat8+                  &               &               &               & $0.11$        &               \\
                           &               &               &               & $(0.07)$      &               \\
lot\_cat5050-6650          &               &               &               &               & $0.04$        \\
                           &               &               &               &               & $(0.03)$      \\
lot\_cat8060+              &               &               &               &               & $0.03$        \\
                           &               &               &               &               & $(0.05)$      \\
\hline
R$^2$                      & $0.65$        & $0.65$        & $0.65$        & $0.65$        & $0.67$        \\
Adj. R$^2$                 & $0.65$        & $0.65$        & $0.65$        & $0.65$        & $0.66$        \\
Num. obs.                  & $1862$        & $1862$        & $1862$        & $1862$        & $1454$        \\
\hline
\multicolumn{6}{l}{\scriptsize{$^{***}p<0.001$; $^{**}p<0.01$; $^{*}p<0.05$}}
\end{tabular}
\caption{Dollar Value of Home Prices}
\label{tab:cat_comp}
\end{center}
\end{table}


\section{Nonlinear Model Specifications}
\subsection{Quadratic Specification for lot size}

%\pagebreak
\section{Example with a Categorical Numerical Variable}

I will revisit the recommended linear model
from Problem Set \#6, 
augmented with a quadratic specification for lot size.
This allowed for an increasing relationship 
between price and lot size, 
for Homes with low lot size, 
but a decreasing relationship for the tractors with high lot size. 
%
In doing so, I will further investigate nonlinear relationships
by incorporating another nonlinear but parametric specification
for the value of lot size. 
This parametric analysis will be performed
using the Box-Tidwell framework
to investigate whether the value of these characteristics
are best described with parametric nonlinear forms. 

%%%%%%%%%%%%%%%%%%%%%%%%%%%%%%%%%%%%%%%%
\clearpage
\section{Linear Regression Model}
%%%%%%%%%%%%%%%%%%%%%%%%%%%%%%%%%%%%%%%%

A natural staring point is the recommended linear model
from Problem Set \#6, augmented with 
the quadratic specification for lot size. 

\subsection{Quadratic Specification for Lot Size}

In the demo for Problem Set \#6, 
we considered the advice of
a used tractor dealer who reported that overpowered used tractors are hard to sell, since they consume more fuel. 
This implies that tractor prices often increase with lot size, up to a point, but beyond that they decrease. 
To incorporate this advice, I created and included a variable for squared lot size. 
A decreasing relationship for high values of lot size
is characterized by 
a positive coefficient on the lot size variable and
a negative coefficient on the squared lot size variable. 

% 

\begin{table}
\begin{center}
\begin{tabular}{l c c c c c}
\hline
 & Model 1 & Model 2 & Model 3 & Model 4 & Model 5 \\
\hline
(Intercept)                & $12.21521^{***}$ & $12.16200^{***}$ & $12.06040^{***}$ & $12.11585^{***}$ & $12.22161^{***}$ \\
                           & $(0.06433)$      & $(0.06661)$      & $(0.08494)$      & $(0.06955)$      & $(0.04377)$      \\
as.factor(num\_beds)2      & $0.08297^{**}$   & $0.06977^{*}$    & $0.06338^{*}$    & $0.07027^{*}$    & $0.07708^{*}$    \\
                           & $(0.03183)$      & $(0.03213)$      & $(0.03228)$      & $(0.03171)$      & $(0.03154)$      \\
as.factor(num\_beds)3      & $0.20487^{***}$  & $0.19005^{***}$  & $0.18141^{***}$  & $0.19585^{***}$  & $0.20529^{***}$  \\
                           & $(0.04011)$      & $(0.04060)$      & $(0.04081)$      & $(0.03879)$      & $(0.03852)$      \\
as.factor(num\_beds)4      & $0.41779^{***}$  & $0.41201^{***}$  & $0.40457^{***}$  & $0.42722^{***}$  & $0.43580^{***}$  \\
                           & $(0.04733)$      & $(0.04742)$      & $(0.04754)$      & $(0.04317)$      & $(0.04298)$      \\
as.factor(num\_beds)6      & $0.47769^{***}$  & $0.47122^{***}$  & $0.48108^{***}$  & $0.52910^{***}$  & $0.52126^{***}$  \\
                           & $(0.07916)$      & $(0.07923)$      & $(0.07934)$      & $(0.06717)$      & $(0.06710)$      \\
as.factor(num\_beds)8      & $0.83073^{***}$  & $0.82682^{***}$  & $0.85696^{***}$  & $0.92787^{***}$  & $0.90046^{***}$  \\
                           & $(0.09758)$      & $(0.09756)$      & $(0.09873)$      & $(0.07654)$      & $(0.07530)$      \\
as.factor(num\_baths)2     & $0.07490^{***}$  & $0.07468^{***}$  & $0.07439^{***}$  & $0.07970^{***}$  & $0.08024^{***}$  \\
                           & $(0.01662)$      & $(0.01664)$      & $(0.01663)$      & $(0.01596)$      & $(0.01597)$      \\
as.factor(num\_baths)3     & $0.08553^{*}$    & $0.08607^{**}$   & $0.08890^{**}$   & $0.09920^{**}$   & $0.09679^{**}$   \\
                           & $(0.03330)$      & $(0.03336)$      & $(0.03337)$      & $(0.03212)$      & $(0.03212)$      \\
floor\_space               & $0.00007$        & $0.00006$        & $0.00006$        &                  &                  \\
                           & $(0.00005)$      & $(0.00005)$      & $(0.00005)$      &                  &                  \\
lot\_size                  & $0.00003^{***}$  & $0.00003^{***}$  & $0.00007^{***}$  & $0.00007^{***}$  & $0.00003^{***}$  \\
                           & $(0.00000)$      & $(0.00000)$      & $(0.00002)$      & $(0.00002)$      & $(0.00000)$      \\
has\_garage                & $0.27672^{***}$  & $0.27667^{***}$  & $0.26832^{***}$  & $0.29584^{***}$  & $0.30557^{***}$  \\
                           & $(0.03296)$      & $(0.03296)$      & $(0.03322)$      & $(0.02275)$      & $(0.02222)$      \\
has\_encl\_patio           & $0.04608^{**}$   & $0.04636^{**}$   & $0.04609^{**}$   & $0.04625^{**}$   & $0.04653^{**}$   \\
                           & $(0.01436)$      & $(0.01439)$      & $(0.01438)$      & $(0.01438)$      & $(0.01439)$      \\
has\_security\_gate        & $0.22997^{***}$  & $0.23140^{***}$  & $0.23056^{***}$  & $0.22970^{***}$  & $0.23051^{***}$  \\
                           & $(0.02056)$      & $(0.02082)$      & $(0.02081)$      & $(0.02080)$      & $(0.02081)$      \\
has\_pool                  & $0.09967^{***}$  & $0.09999^{***}$  & $0.10085^{***}$  & $0.10102^{***}$  & $0.10016^{***}$  \\
                           & $(0.02560)$      & $(0.02558)$      & $(0.02557)$      & $(0.02557)$      & $(0.02559)$      \\
transit\_score             & $0.06665^{***}$  & $0.06650^{***}$  & $0.06676^{***}$  & $0.06687^{***}$  & $0.06661^{***}$  \\
                           & $(0.00346)$      & $(0.00346)$      & $(0.00346)$      & $(0.00346)$      & $(0.00346)$      \\
school\_score              & $0.00405$        &                  &                  &                  &                  \\
                           & $(0.00327)$      &                  &                  &                  &                  \\
type\_of\_buyerRental      & $0.11080^{***}$  & $0.11148^{***}$  & $0.11205^{***}$  & $0.11083^{***}$  & $0.11019^{***}$  \\
                           & $(0.01879)$      & $(0.01884)$      & $(0.01882)$      & $(0.01879)$      & $(0.01881)$      \\
age                        & $-0.00974^{***}$ & $-0.00968^{***}$ & $-0.00967^{***}$ & $-0.00965^{***}$ & $-0.00967^{***}$ \\
                           & $(0.00048)$      & $(0.00048)$      & $(0.00048)$      & $(0.00048)$      & $(0.00048)$      \\
as.factor(school\_score)2  &                  & $0.10288^{**}$   & $0.10070^{**}$   & $0.10145^{**}$   & $0.10371^{**}$   \\
                           &                  & $(0.03325)$      & $(0.03325)$      & $(0.03325)$      & $(0.03325)$      \\
as.factor(school\_score)3  &                  & $0.10013^{**}$   & $0.09819^{**}$   & $0.09895^{**}$   & $0.10095^{**}$   \\
                           &                  & $(0.03231)$      & $(0.03230)$      & $(0.03230)$      & $(0.03231)$      \\
as.factor(school\_score)4  &                  & $0.09229^{**}$   & $0.08833^{**}$   & $0.08920^{**}$   & $0.09327^{**}$   \\
                           &                  & $(0.03177)$      & $(0.03181)$      & $(0.03181)$      & $(0.03176)$      \\
as.factor(school\_score)5  &                  & $0.08607^{**}$   & $0.08086^{*}$    & $0.08133^{*}$    & $0.08665^{**}$   \\
                           &                  & $(0.03187)$      & $(0.03196)$      & $(0.03196)$      & $(0.03187)$      \\
as.factor(school\_score)6  &                  & $0.10210^{**}$   & $0.09705^{**}$   & $0.09803^{**}$   & $0.10320^{**}$   \\
                           &                  & $(0.03364)$      & $(0.03372)$      & $(0.03371)$      & $(0.03363)$      \\
as.factor(school\_score)7  &                  & $0.09600^{**}$   & $0.09074^{**}$   & $0.09105^{**}$   & $0.09641^{**}$   \\
                           &                  & $(0.03266)$      & $(0.03275)$      & $(0.03275)$      & $(0.03266)$      \\
as.factor(school\_score)8  &                  & $0.09904^{**}$   & $0.09437^{**}$   & $0.09534^{**}$   & $0.10013^{**}$   \\
                           &                  & $(0.03277)$      & $(0.03283)$      & $(0.03282)$      & $(0.03276)$      \\
as.factor(school\_score)9  &                  & $0.08716^{*}$    & $0.08297^{*}$    & $0.08407^{*}$    & $0.08837^{*}$    \\
                           &                  & $(0.03503)$      & $(0.03507)$      & $(0.03506)$      & $(0.03502)$      \\
as.factor(school\_score)10 &                  & $0.09569^{*}$    & $0.10108^{*}$    & $0.10230^{*}$    & $0.09688^{*}$    \\
                           &                  & $(0.04156)$      & $(0.04162)$      & $(0.04161)$      & $(0.04155)$      \\
sq\_lot\_size              &                  &                  & $-0.00000$       & $-0.00000$       &                  \\
                           &                  &                  & $(0.00000)$      & $(0.00000)$      &                  \\
\hline
R$^2$                      & $0.65037$        & $0.65268$        & $0.65338$        & $0.65314$        & $0.65242$        \\
Adj. R$^2$                 & $0.64714$        & $0.64795$        & $0.64847$        & $0.64842$        & $0.64787$        \\
Num. obs.                  & $1862$           & $1862$           & $1862$           & $1862$           & $1862$           \\
\hline
\multicolumn{6}{l}{\scriptsize{$^{***}p<0.001$; $^{**}p<0.01$; $^{*}p<0.05$}}
\end{tabular}
\caption{Quadratic Models for Home Prices}
\label{tab:reg_sq_lot}
\end{center}
\end{table}

% 
The results of this regression specification are shown in 
Table \ref{tab:reg_sq_lot}. 
%
The squared lot size variable has a coefficient of $-2.081e-05$, which is nearly ten times as large as the standard error of $2.199e-06$, which is very strong evidence against the null hypothesis of a positive or zero coefficient. 
I conclude that the log of the sale price does decline for large values of lot size. 


With the squared lot size variable, the $\bar{R}^2$ is $0.764$, indicating that it is a much stronger model than the others we considered. 
The $F$-statistic is large, indicating that it is a better candidate than the simple average log sale price. 
The new squared lot size variable is statistically significant and the theory behind it is sound, since above a certain point, added lot size may not improve performance but will cost more to operate. 
This new model is much improved over the previous models with a linear specification for lot size.
Next, I will attempt to improve on this specification, 
as we did for Problem Set \#8. 





%%%%%%%%%%%%%%%%%%%%%%%%%%%%%%%%%%%%%%%%
% \clearpage
\section{Nonlinear Specifications}
%%%%%%%%%%%%%%%%%%%%%%%%%%%%%%%%%%%%%%%%


%\pagebreak
\subsection{The Box--Tidwell Transformation}

The Box--Tidwell function tests for non-linear relationships
to the mean of the dependent variable.
The nonlinearity is in the form of an
exponential transformation in the form of the Box-Cox
transformation, except that the transformation is taken
on the explanatory variables.


\subsubsection{Transformation of lot size}


Performing the transformation on the lot size variable
produces a modified form of the linear model.
This specification allows a single exponential
transformation on lot size, rather than a quadratic form.

\begin{verbatim} MLE of lambda Score Statistic (z) Pr(>|z|)  
       0.13681             -2.0662  0.03881 *
---
Signif. codes:  0 '***' 0.001 '**' 0.01 '*' 0.05 '.' 0.1 ' ' 1

iterations =  7 
\end{verbatim}

The \textsf{R} output is the statistics for a test of nonlinearity:
that the exponent $\lambda$ in the Box--Tidwell transformation is zero.
%
The "\texttt{MLE of lambda}" statistic is the optimal exponent on lot size.
Similar to the Box-Cox transformation,
with Box-Tidwell, the exponents are on the explanatory variables
and are all called lambda, in contrast
to the parameter $\tau$ in our class notes.
The exponent is significantly different from 0,
although it is a small positive value,
which suggests an increasing relationship
for the value of lot size
with a slope that is sharply declining.
Next I consider the possibility of a changing relationship 
for the next continuous variable. 


\subsubsection{Transformation of Age}


\begin{verbatim} MLE of lambda Score Statistic (z) Pr(>|z|)  
       0.69941              2.3896  0.01687 *
---
Signif. codes:  0 '***' 0.001 '**' 0.01 '*' 0.05 '.' 0.1 ' ' 1

iterations =  3 
\end{verbatim}

This coefficient is effectively 1, which is more evidence of
a purely linear relationship between \texttt{log\_saleprice}
and age: the percentage depreciation rate is constant.
Next, I will consider the possibility of nonlinearity 
in depreciation from hours of use. 


\subsubsection{Transformation of All Three Continuous Variables}


\begin{verbatim}         MLE of lambda Score Statistic (z) Pr(>|z|)  
lot_size      0.082757             -2.0269  0.04267 *
age           0.697634              2.3932  0.01670 *
---
Signif. codes:  0 '***' 0.001 '**' 0.01 '*' 0.05 '.' 0.1 ' ' 1

iterations =  8 
\end{verbatim}


The performance is similar to the other models with
forms of nonlinearity for the value of lot size.
Now consider the full set of such models in a table for a final comparison.


\pagebreak
\section{Final Comparison of Candidate Models}

I created a variable \texttt{lot size\_bt}
by raising lot size to the optimal exponent 
$\hat{\lambda} = 0.1143693$. 
Then, I included this variable in the place of 
the lot size variables a the linear regression model.
% 
Table \ref{tab:reg_sq_lot_bt} collects the results
of the set of models from the three forms of nonlinearity.
Model 1 is the linear regression model with 
a quadratic form for lot size. 
Model 2 
has the same specification as the other one, 
except that the lot size variable is transformed using the optimal
exponent for the Box-Tidwell transformation. 
% 
The last model has the highest R-squared
among the ones we have estimated.
Again, the differences are marginal, so the practical recommendation
is the model with the quadratic relationship for lot size, 
which has a simpler interpretation.
In either case, we conclude that John Deere tractors are worth
approximately thirty percent more valuable
than an equivalent tractor of another brand. 
Compare this with the lower premium of 17\%, 
which was not even statistically significant, 
when we estimated a simpler, linear specification
in which we ignored the nonlinearity in the model. 


\begin{table}
\begin{center}
\begin{tabular}{l c c}
\hline
 & Model 1 & Model 2 \\
\hline
(Intercept)                & $12.11585^{***}$ & $10.86297^{***}$ \\
                           & $(0.06955)$      & $(0.21050)$      \\
as.factor(num\_beds)2      & $0.07027^{*}$    & $0.07009^{*}$    \\
                           & $(0.03171)$      & $(0.03159)$      \\
as.factor(num\_beds)3      & $0.19585^{***}$  & $0.19584^{***}$  \\
                           & $(0.03879)$      & $(0.03862)$      \\
as.factor(num\_beds)4      & $0.42722^{***}$  & $0.42747^{***}$  \\
                           & $(0.04317)$      & $(0.04297)$      \\
as.factor(num\_beds)6      & $0.52910^{***}$  & $0.52439^{***}$  \\
                           & $(0.06717)$      & $(0.06692)$      \\
as.factor(num\_beds)8      & $0.92787^{***}$  & $0.91419^{***}$  \\
                           & $(0.07654)$      & $(0.07482)$      \\
as.factor(num\_baths)2     & $0.07970^{***}$  & $0.08001^{***}$  \\
                           & $(0.01596)$      & $(0.01595)$      \\
as.factor(num\_baths)3     & $0.09920^{**}$   & $0.09896^{**}$   \\
                           & $(0.03212)$      & $(0.03206)$      \\
lot\_size                  & $0.00007^{***}$  &                  \\
                           & $(0.00002)$      &                  \\
sq\_lot\_size              & $-0.00000$       &                  \\
                           & $(0.00000)$      &                  \\
has\_garage                & $0.29584^{***}$  & $0.29816^{***}$  \\
                           & $(0.02275)$      & $(0.02246)$      \\
has\_encl\_patio           & $0.04625^{**}$   & $0.04614^{**}$   \\
                           & $(0.01438)$      & $(0.01438)$      \\
has\_security\_gate        & $0.22970^{***}$  & $0.22968^{***}$  \\
                           & $(0.02080)$      & $(0.02077)$      \\
has\_pool                  & $0.10102^{***}$  & $0.10156^{***}$  \\
                           & $(0.02557)$      & $(0.02557)$      \\
transit\_score             & $0.06687^{***}$  & $0.06684^{***}$  \\
                           & $(0.00346)$      & $(0.00346)$      \\
as.factor(school\_score)2  & $0.10145^{**}$   & $0.10112^{**}$   \\
                           & $(0.03325)$      & $(0.03322)$      \\
as.factor(school\_score)3  & $0.09895^{**}$   & $0.09932^{**}$   \\
                           & $(0.03230)$      & $(0.03228)$      \\
as.factor(school\_score)4  & $0.08920^{**}$   & $0.08966^{**}$   \\
                           & $(0.03181)$      & $(0.03176)$      \\
as.factor(school\_score)5  & $0.08133^{*}$    & $0.08141^{*}$    \\
                           & $(0.03196)$      & $(0.03188)$      \\
as.factor(school\_score)6  & $0.09803^{**}$   & $0.09800^{**}$   \\
                           & $(0.03371)$      & $(0.03365)$      \\
as.factor(school\_score)7  & $0.09105^{**}$   & $0.09175^{**}$   \\
                           & $(0.03275)$      & $(0.03269)$      \\
as.factor(school\_score)8  & $0.09534^{**}$   & $0.09602^{**}$   \\
                           & $(0.03282)$      & $(0.03278)$      \\
as.factor(school\_score)9  & $0.08407^{*}$    & $0.08538^{*}$    \\
                           & $(0.03506)$      & $(0.03501)$      \\
as.factor(school\_score)10 & $0.10230^{*}$    & $0.10047^{*}$    \\
                           & $(0.04161)$      & $(0.04125)$      \\
type\_of\_buyerRental      & $0.11083^{***}$  & $0.11068^{***}$  \\
                           & $(0.01879)$      & $(0.01879)$      \\
age                        & $-0.00965^{***}$ & $-0.00964^{***}$ \\
                           & $(0.00048)$      & $(0.00048)$      \\
lot\_bt                    &                  & $0.47925^{***}$  \\
                           &                  & $(0.06601)$      \\
\hline
R$^2$                      & $0.65314$        & $0.65315$        \\
Adj. R$^2$                 & $0.64842$        & $0.64862$        \\
Num. obs.                  & $1862$           & $1862$           \\
\hline
\multicolumn{3}{l}{\scriptsize{$^{***}p<0.001$; $^{**}p<0.01$; $^{*}p<0.05$}}
\end{tabular}
\caption{Alternate Models for Home Prices}
\label{tab:reg_sq_lot_bt}
\end{center}
\end{table}


Table \ref{tab:reg_age_bt} compare previours model with variable age transformed using $\hat{\lambda} = 0.1143693$. 


\begin{table}
\begin{center}
\begin{tabular}{l c c}
\hline
 & Model 1 & Model 2 \\
\hline
(Intercept)                & $10.86297^{***}$ & $10.90045^{***}$ \\
                           & $(0.21050)$      & $(0.20893)$      \\
lot\_bt                    & $0.47925^{***}$  & $0.47974^{***}$  \\
                           & $(0.06601)$      & $(0.06549)$      \\
as.factor(num\_beds)2      & $0.07009^{*}$    & $0.07393^{*}$    \\
                           & $(0.03159)$      & $(0.03135)$      \\
as.factor(num\_beds)3      & $0.19584^{***}$  & $0.20211^{***}$  \\
                           & $(0.03862)$      & $(0.03832)$      \\
as.factor(num\_beds)4      & $0.42747^{***}$  & $0.43336^{***}$  \\
                           & $(0.04297)$      & $(0.04263)$      \\
as.factor(num\_beds)6      & $0.52439^{***}$  & $0.53277^{***}$  \\
                           & $(0.06692)$      & $(0.06638)$      \\
as.factor(num\_beds)8      & $0.91419^{***}$  & $0.92310^{***}$  \\
                           & $(0.07482)$      & $(0.07425)$      \\
as.factor(num\_baths)2     & $0.08001^{***}$  & $0.07966^{***}$  \\
                           & $(0.01595)$      & $(0.01582)$      \\
as.factor(num\_baths)3     & $0.09896^{**}$   & $0.09720^{**}$   \\
                           & $(0.03206)$      & $(0.03181)$      \\
has\_garage                & $0.29816^{***}$  & $0.29609^{***}$  \\
                           & $(0.02246)$      & $(0.02229)$      \\
has\_encl\_patio           & $0.04614^{**}$   & $0.04641^{**}$   \\
                           & $(0.01438)$      & $(0.01427)$      \\
has\_security\_gate        & $0.22968^{***}$  & $0.22896^{***}$  \\
                           & $(0.02077)$      & $(0.02061)$      \\
has\_pool                  & $0.10156^{***}$  & $0.09923^{***}$  \\
                           & $(0.02557)$      & $(0.02537)$      \\
transit\_score             & $0.06684^{***}$  & $0.06675^{***}$  \\
                           & $(0.00346)$      & $(0.00343)$      \\
as.factor(school\_score)2  & $0.10112^{**}$   & $0.10255^{**}$   \\
                           & $(0.03322)$      & $(0.03296)$      \\
as.factor(school\_score)3  & $0.09932^{**}$   & $0.09679^{**}$   \\
                           & $(0.03228)$      & $(0.03203)$      \\
as.factor(school\_score)4  & $0.08966^{**}$   & $0.08604^{**}$   \\
                           & $(0.03176)$      & $(0.03152)$      \\
as.factor(school\_score)5  & $0.08141^{*}$    & $0.07768^{*}$    \\
                           & $(0.03188)$      & $(0.03163)$      \\
as.factor(school\_score)6  & $0.09800^{**}$   & $0.09447^{**}$   \\
                           & $(0.03365)$      & $(0.03338)$      \\
as.factor(school\_score)7  & $0.09175^{**}$   & $0.09305^{**}$   \\
                           & $(0.03269)$      & $(0.03243)$      \\
as.factor(school\_score)8  & $0.09602^{**}$   & $0.09403^{**}$   \\
                           & $(0.03278)$      & $(0.03253)$      \\
as.factor(school\_score)9  & $0.08538^{*}$    & $0.08368^{*}$    \\
                           & $(0.03501)$      & $(0.03473)$      \\
as.factor(school\_score)10 & $0.10047^{*}$    & $0.10047^{*}$    \\
                           & $(0.04125)$      & $(0.04093)$      \\
type\_of\_buyerRental      & $0.11068^{***}$  & $0.11198^{***}$  \\
                           & $(0.01879)$      & $(0.01864)$      \\
age                        & $-0.00964^{***}$ &                  \\
                           & $(0.00048)$      &                  \\
age\_bt                    &                  & $-0.03120^{***}$ \\
                           &                  & $(0.00150)$      \\
\hline
R$^2$                      & $0.65315$        & $0.65856$        \\
Adj. R$^2$                 & $0.64862$        & $0.65409$        \\
Num. obs.                  & $1862$           & $1862$           \\
\hline
\multicolumn{3}{l}{\scriptsize{$^{***}p<0.001$; $^{**}p<0.01$; $^{*}p<0.05$}}
\end{tabular}
\caption{Alternate Models for Home Prices}
\label{tab:reg_sq_age_bt}
\end{center}
\end{table}

%%%%%%%%%%%%%%%%%%%%%%%%%%%%%%%%%%%%%%%%
\end{document}
%%%%%%%%%%%%%%%%%%%%%%%%%%%%%%%%%%%%%%%%
